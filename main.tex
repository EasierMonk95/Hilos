% ******************************* Plantilla para informes de laboratorio para la Escuela de Ciencia e Ingenieria en Materiales **************************
%UdeA
\documentclass[letterpaper]{article}


%%%%%%%%%%%%
%Paquetes
%%%%%%%%%%%%
%\documentclass[journal=esthag,manuscript=article]{achemso}
\usepackage[spanish]{babel}
\usepackage[utf8]{inputenc}
\usepackage{fancyhdr}
\usepackage{authblk}
\usepackage{amsfonts}
\usepackage[version=3]{mhchem}
\usepackage{textgreek}
\usepackage{pdflscape} % landscape pages
\usepackage{lmodern} % Previene advertencias sobre fuentes no disponibles
\usepackage{graphicx}
\usepackage[left=3cm,right=2cm,top=2cm,bottom=3cm]{geometry} % Configura los margenes de página
%\usepackage{lineno} %Numeros en las lineas
%\linenumbers
%Paquete para incluir hipervínculos
\usepackage[colorlinks=true, 
linkcolor = black,
urlcolor  = blue,
citecolor = black,
anchorcolor = blue]{hyperref}
\usepackage{lipsum}
\usepackage{Sweave}
\usepackage{multirow}
\usepackage[table,xcdraw]{xcolor}
\usepackage{array}
\usepackage{tabu}
\usepackage{caption} %Permite escribir en negrita los titulos de las figuras y las tablas.
\usepackage[labelfont=bf]{caption} %Configura los titulos de las figuras y las tablas en negrita.

%Paquete para el manejo de las unidades en el Sistema Internacional
\usepackage{siunitx}
\sisetup{mode=text, output-decimal-marker = {,}, per-mode = symbol, qualifier-mode = phrase, qualifier-phrase = { de }, list-units = brackets, range-units = brackets, range-phrase = --}
\DeclareSIUnit[number-unit-product = \;] \atmosphere{atm}
\DeclareSIUnit[number-unit-product = \;] \pound{lb}
\DeclareSIUnit[number-unit-product = \;] \inch{"}
\DeclareSIUnit[number-unit-product = \;] \foot{ft}
\DeclareSIUnit[number-unit-product = \;] \yard{yd}
\DeclareSIUnit[number-unit-product = \;] \mile{mi}
\DeclareSIUnit[number-unit-product = \;] \pint{pt}
\DeclareSIUnit[number-unit-product = \;] \quart{qt}
\DeclareSIUnit[number-unit-product = \;] \flounce{fl-oz}
\DeclareSIUnit[number-unit-product = \;] \ounce{oz}
\DeclareSIUnit[number-unit-product = \;] \degreeFahrenheit{\SIUnitSymbolDegree F}
\DeclareSIUnit[number-unit-product = \;] \degreeRankine{\SIUnitSymbolDegree R}
\DeclareSIUnit[number-unit-product = \;] \usgallon{galón}
\DeclareSIUnit[number-unit-product = \;] \uma{uma}
\DeclareSIUnit[number-unit-product = \;] \ppm{ppm}
\DeclareSIUnit[number-unit-product = \;] \eqg{eq-g}
\DeclareSIUnit[number-unit-product = \;] \normal{\eqg\per\liter\of{solución}}
\DeclareSIUnit[number-unit-product = \;] \molal{\mole\per\kilo\gram\of{solvente}}


\addto\captionsspanish{\renewcommand{\tablename}{Tabla}} %Cambia el nombre de cuadros a tablas
%%%%%%%%%%%%%
%Encabezado
%%%%%%%%%%%%%  
\title{\textbf{Interrupciones y sus finalidades}}
\author{Simón Sánchez Rúa}
\affil{Facultad de Ingeniería Univesridad de Antioquia}


%Imagenes en el encabezado de página

\rhead{\begin{picture}(0,0) \put(-86.7,0){\includegraphics[width=30mm]{./images/UDEA.png}} \end{picture}}
\renewcommand{\headrulewidth}{0.5pt}
\pagestyle{fancy}

%%%%%%%%%%%%%%%%%%%%%%%
%Inicio del documento
%%%%%%%%%%%%%%%%%%%%%%%
\begin{document}
	
	
%\input{2015-05-30_PEEC_trip_report-concordance}
\maketitle
\renewenvironment{abstract}
 {\quotation\small\noindent\rule{\linewidth}{.5pt}\par\smallskip
  {\centering\bfseries\abstractname\par}\medskip}
 {\par\noindent\rule{\linewidth}{.5pt}\endquotation}


%%%%%%%%%%%%%%%
%Hilos
%%%%%%%%%%%%%%%
\section{Hilos}

Los hilos son parte de los núcleos de un procesador, estos son los "canales" por los cuales pasa la información para ser procesada, para realizar una ejecución de alguna función de algún programa se debe apilar el dato a trabajar en un hilo y esperar a que se procesen todos los datos anteriores para que este reciba su turno, sin embargo la finalidad de los hilos es poder optimizar la ejecución de tareas, por eso ahora en la actualidad aunque como mínimo en cada núcleo hay un hilo también puede haber más de un hilo por núcleo, lo que hace que en vez de que los datos deban esperar apilados por su turno, puedan ser procesados parcialmente por el mismo núcleo mientras se accede a algún requerimiento de la función que ya estaba siendo procesada, es decir, con los hilos puede pasar más de un dato por el procesador en caso necesario de detener el flujo de datos en uno de los hilos por necesidades como acceder a la memoria o atender una interrupción, de esa forma se hace que un procesador sea multitarea y ahorre tiempo de ejecución dando mejor uso a los núcleos para que se avance temporalmente con varios datos a la vez puesto que de esa forma de procesan datos de manera intercalada y así la "fila" de información fluye más rápidamente en caso de que uno de los hilos deba detener una ejecución o tenga un programa muy complejo siendo procesado y así poder procesar otros programas más simples mientras se intercala con el programa pesado para poder ahorrar tiempos de procesamiento y ejecución.

\subsection{¿Historia?}

Los hilos son como tal parte de los núcleos en la actualidad, pero al inicio su introducción se dió con el nombre de "tareas" en 1967 como parte de un Sistema Operativo llamado MVT, pero además de eso solo se sabe que de ahí en adelante cambiaron de nombre y se transformaron en mucho más esenciales que al inicio porque ahora además de venir incorporados en cada núcleo del procesador, también su creación dió pie al uso de varios en un mismo núcleo para cumplir lo explicado anteriormente siendo esto multi-hilos o multitareas.

%%%%%%%%%%%%%%%%%%%%%%%
%Tipos de hilos
%%%%%%%%%%%%%%%%%%%%%%%
\section{Tipos de hilos}

A lo que se refiere a los tipos de hilos tenemos presentes dos tipos que son enfocados de maneras muy diferentes, uno al mundo físico y el otro al digital.


\subsection{Hilos de kernel}
Este tipo de hilos son los predispuestos por el Sistema Operativo a ser manejados, estos son programados y predispuestos por los mismos desarroladores del procesador y generalmente tienen funciones de multitareas lo suficientemente necesarias para poder ejecutar varios programas que se instalen a un computador o célular sin problemas de ralentización, estos se limitan por la cantidad de hilos presentes en cada núcleo y mínimo habrá un hilo presente por núcleo, sin embargo no porque haya un solo hilo significa que el procesador será lento, puesto que para eso se creó el uso de muchos núcleos y los multi-hilos es para agilizar los procesos y poder así terminar o ejecutar ciertos programas rápidamente para simular que hay varias tareas a las vez.



\subsection{Hilos del usuario}

Estos hilos son fáciles de crear y destruir, puesto que estos son programados por el usuario, son iguales que cualquier otro hilo (tienen los mismos usos y finalidades) y se puedene hacer cuantos se quiera, la idea de hacer los hilos de esta manera es poder condicionar el flujo de datos, orden de procesos o tareas, destruirlos o crearlos facilmente, etc. Todas esas funciones sin embargo no pueden hacerse de la nada, puesto que igual estos aunque no dependen del kernel para una cantidad, igual dependen físicamente del procesador para no sobrecargar el computador, por lo que al hacer hilos por usuario se debe tener en cuenta ciertas relaciones entre hilos de usuario y kernel como:

-Many to One (Muchos por uno): Este tipo de relación lo que hace es asignar varios hilos a un hilo de kernel, sin embargo si uno de estos se bloquea toda la ejecución de sebe bloquear porque solo hay una vía de información.

-One to One (Uno a Uno): Estos a diferencia del anterior se basan en asiganar un hilo de usuario a uno de kernel, sin embargo esto hace que no se bloquee toda la ejecución en un hilo porque se puede seguir con otro en esos casos, sin embargo cada vez que se crea un hilo de usuario se asigna a uno de kernel y los últimos están limitados en cantidad por el sistema.

-Many to Many (muchos para muchos): Este modelo se basa en asignar varios hilos de usuario a conjuntos de hilos de kernel, por lo tanto los procesosse reparten en el kernel y si uno de estos se bloquea se puede hacer que otro hilo se encargue del programa, como se usan conjuntos de hilos para encargarse de varios hilos de usuario no hay un límite para crear hilos por parte del usuario y de esa forma se usan las mejores características de los anteriores modelos.

\section{Implementaciones}
Para el uso de hilos estos se pueden implementar de dos formas.

\subsection{Implementación por Hardware}
Los hilos a nivel de Hardware se implementan por parte de los desarrolladores de los procesadores, ellos definen la cantidad de hilos que tendrá su procesador (teniendo en cuenta lo posible físicamente a través de estudios y pruebas con habilidades de ingenieros en varias áreas) y la forma en la que trabajan, si se es un ingeniero y se tiene habilidades necesarias para diseñar un procesador, se podrá implementar hilos en el mismo si queremos, generalmente se implementan en el procesador con usos de ejecución de programas y tareas básicas, en resumen los añaden cantidades dependiendo del sistema y su finalidad planteada, y los programan con funciones generales para que abarquen la mayor cantidad de usos posibles de manera eficaz cumpliendo con los deseos de la mayoría de usuarios. 

\subsection{Implementación de Software}
Para implementar los hilos por Software se debe programarlos mediante el uso de una biblioteca que trae las funciones necesarias para crear un hilo, destruirlo, definir sus planificaciones, etc, cada uno de los usos de hilos por Software está ligado explicitamente con el uso de código, ya que es a través de la programación que estos toman "vida", por lo tanto sus usos que son varios todos se definen con lenguajes de programación y dependen de como se definan para que sus usos se liguen con los del hardware, esto hace que cada persona pueda programar sus procesos y decidir las formas en que sus procesadores ejecutan programas y manejan datos a su antojo. Los hilos de Software son los de usuario y se implementan por el mismo como quiera y si lo quiere, puesto que no es necesario ya que estos se implementan en casos muy especiales o necesidades claras y definidas, ya que los hilos que vienen predeterminados ya traen consigo lo mejor para hacer efectivo el procesamiento de información cuando un usuario no necesita ordenes específicas de ejecución de programas o de planificación de manejo de estructuras de datos, esta implementación es opcional y si la quieres realizar requerirás de programación para hacerlo.


%%%%%%%%%%%%%%%
% Referencias
%%%%%%%%%%%%%%%
\bibliographystyle{plain}

\begin{thebibliography}{9}


\bibitem{Wiki} Harris, G.(2019). QWE: Hilo(Informática). Recuperado de \url{https://es.qwe.wiki/wiki/Thread_(computing)#Processes,_kernel_threads,_user_threads,_and_fibers}

\bibitem{Blogger} HILOS EN SISTEMAS OPERATIVOS (2013). Recuperado de \url{http://resenadelossistemasoperativos.blogspot.com/2013/09/hilos-en-sistemas-operativos.html}

\bibitem{video1} Martinez, P.[Informática pa novatos]. (2018, mayo, 21). Procesadores, cores, hilos y MAS en Español!![Archivo de Vídeo]. Recuperado de: \url{https://www.youtube.com/watch?v=RGrY62314Go&t=587s}

\bibitem{video2} Jonatan. [Entorno Simple]. (2019, abril, 11). LOS HILOS DEL PROCESADOR - #ESimple [Archivo de video]. Recuperado de: \url{https://www.youtube.com/watch?v=v2UH0DmwxXk&t=5s}

\end{thebibliography}

\end{document}
